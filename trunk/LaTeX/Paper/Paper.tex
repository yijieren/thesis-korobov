
% Created 2011-10-27 do 16:51
\documentclass[a4paper, 12pt]{article}
\usepackage[utf8]{inputenc}
\usepackage[T1]{fontenc}
\usepackage{fixltx2e}
\usepackage{graphicx}
\usepackage{longtable}
\usepackage{float}
\usepackage{wrapfig}
\usepackage{soul}
\usepackage{textcomp}
%\usepackage{marvosym}
%\usepackage{wasysym}
%\usepackage{latexsym}
%\usepackage{amssymb}
\usepackage{amsmath, amsthm, amssymb}
\usepackage{mathtools}
\usepackage{hyperref}
\usepackage{placeins}
\usepackage{pict2e}
\usepackage{subfig}
\usepackage{color}
\usepackage{multirow}
\usepackage[usenames,dvipsnames,svgnames,table]{xcolor}
%\usepackage{cmbright}
\usepackage[english]{babel}
\usepackage[a4paper,margin=2.5cm]{geometry}
 \hyphenpenalty=5000
\tolerance=1000
\providecommand{\alert}[1]{\textbf{#1}}
\newfloat{MATLAB code}{h}{}
\usepackage{tikz,pgfplots}

\pgfplotsset{compat=newest}
\pgfplotsset{plot coordinates/math parser=false}

\newtheorem{mydef}{THEOREM}

\title{Beschrijving Algoritmes}
\author{Roel Matthysen - \today}
\date{}

\newlength\figureheight
\newlength\figurewidth
\setlength\figureheight{2cm}
\setlength\figurewidth{12cm}


\begin{document}

\maketitle
The concept of optimal coefficients was introduced in [1], and their significance for the approximate computation of multidimensional integrals of arbitrary multiplicity $s$ was indicated. Various algorithms for computing $s$-dimensional optimal coefficients modulo $p$ where $p$ is he number of nodes of the quadrature formula were obtained in [1]-[3]. The realization of these algorithms required the execution of $O(p^2)$ or $O(p^{1+1/3})$ elementary arithmetic operations.

In this note we present more economical algorithms for $p=2^n$ whose realization requires the execution of $O(p)$ or $O(p\ln{p}$ operations.

Let $n$ and $s$ be positive integers, and $x_1,...,x_s$ odd integers. Summations over odd integers $m$ is indicated by $\sum_m^*$. For $v=1,...,n$ we define the function $h_v(x_1,x_2,...,x_s)$ by
 \begin{equation}
h_v(x_1,x_2,...,x_s)=\frac{1}{2^v}\sum_{m=1}^{2^v}{}^*\left(2n-2v+\frac{1}{||mx_1/2^v||}\right)\cdots\left(2n-2v+\frac{1}{||mx_s/2^v||}\right)
\end{equation}
where $||mx_j/2^v||$ is the distance from $mx_j/2^v$ to the nearest integer. 

Take $a_{11}=...=a_{s1}=1$. Suppose that $v\ge2$ and that the odd integers $a_{1v-1},...,a_{sv-1}$ are known for $2\le v \le n$ we define $a_{1v},...,a_{sv}$ by the equalities
\begin{equation}
a_{1v}=a_{1v-1}+2^{v-1}z_1',...,a_{sv}=a_{sv-1}+2^{v-1}z_s'
\end{equation} 
where $z_1',...z_s'$ are the variables at which the function
\begin{equation}
h_v(a_{1v-1}+2^{v-1}z_1',...,a_{sv-1}+2^{v-1}z_s'
\end{equation}
attains a minimum as the variables $z_1,...,z_s$ run through the values 0 and 1 independently.
\begin{mydef}
For an arbitrary positive integer n the integer $a_1,...,a_s$ defined by the equalities $a_1=a_{1n},...,a_s=a_{sn}$ are optimal coeffecients modulo $p=2^n$.
\end{mydef}
\begin{proof}
For $v=1,...,n$ we introduce the notation
\begin{equation}
h_v=
\end{equation}
\end{proof}
\end{document}

%%% Local Variables: 
%%% mode: latex
%%% TeX-master: t
%%% End: